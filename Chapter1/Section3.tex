\section{問題点と既存の解決策}
Guillaumeらの研究\cite{bellegarda2022cpgrl}では深層強化学習に明示的にCPG (central pattern generator)を組み込んだ手法(以下CPG-RLと呼称)を用いて,通常の深層強化学習に比べてシンプルな報酬関数の設計で四脚ロボットの歩容を生成した.
CPGのフレームワークを用いて学習することで,学習の高速化とSim-to-Realの平易化を実現した.
また,全方向への移動が可能であり,ロボット質量の115%の13.75kgの荷重を動的に追加されても歩行を持続できるなど高いパフォーマンスを発揮した.
しかし,CPG-RL\cite{bellegarda2022cpgrl}では平地のみで学習を行っており,微小な段差へのロバスト性は示されているが,四脚ロボットが期待される不整地の歩行性能には届いていない.
脊椎動物は歩行や前足の配置をどのように調整するかを脊髄上部の領域で決定していること\cite{DREW201525}や代償的姿勢反応では脳幹網様体が関与していると\cite{doi:10.1152/jn.91013.2008}が知られており,姿勢を安定させ,不整地での十分な歩行性能を得るためにはCPGのみでなく,脊髄上部が担うとされる姿勢反射の機能が必要であると考える.
CPG-RLを用いて予期的な運動調整を行たった研究\cite{cpgrl2,cpgrl3}では,視覚情報を用いてCPG-RLのフレームワークで脚軌道中心のオフセットを調整し,ギャップのある地形での歩行を可能にした.

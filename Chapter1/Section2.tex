\section{関連研究}
\subsection{モデルベースのコントローラ}
四脚ロボットのコントローラの一つにモデル予測制御(MPC)\cite{MIT_Cheetah_3,kim2019highly,sombolestan2021adaptive}に代表されるモデルベースのコントローラがある.
これらの研究では実機のロボットにおいて高い性能を示してきた.
しかし,これらの手法はオンライン最適化を行うため,モデルの精度への依存度が高く,緻密な設計が必要である.
また,事前に設計された足先の軌道やヒューリスティックな制御を用いるため,環境の変動に対する適応性が低い.
% Learning Based
\subsection{学習ベースのコントローラ}
一方,深層強化学習などを用いた学習ベースのコントローラは外乱や未知の環境に対してロバストな制御ポリシーを学習することが期待でき,注目を集めている.
近年の研究では、固有感覚のみを用いた「ブラインド」制御ポリシーの学習に成功しており,平地\cite{tan2018simtoreal,Hwangbo_2019}と不整地\cite{kumar2021rma,challenging_terrain,margolis2022walkwaystuningrobot}の両方での制御に成功している.
しかし,得られる動作が不自然であったり,報酬関数の設計が複雑になり,設計者のスキルへの依存が大きくなったりする.
それらを改善するために,実際の動物の運動データをエキスパートとして用いて模倣学習を行った研究\cite{RoboImitationPeng20}などもあるが,事前に膨大なエキスパートデータを必要とする.
% Biology Inspired
\subsection{バイオインスパイアのコントローラ}
また,四脚ロボットの歩行においては生物学から着想を得たバイオインスパイアのコントローラに関する研究も多く存在する.
特に脊椎動物の脊髄に存在する周期的な出力信号の協調パターンを生成する神経回路であるCPG(central pattern generator)に着想を得た研究が多い.
% このへんで改段落してCPGについての話題をまとめてもよさそうです。
CPGと単純な力フィードバックによって四脚ロボットの歩容の生成と遷移を可能にした研究\cite{Owaki1,Owaki2}やCPGにタッチセンサーからの感覚フィードバックを追加した研究\cite{4543306}などCPGを用いた手法はシンプルで汎用的である.